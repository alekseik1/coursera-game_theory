%!TEX root=main.tex
	\section{Модели Курно и Бертрана. Монополистическая конкуренция.}
	
	\subsection{Тренировочные задачи}
	
	\task
	Рассмотрим модель Курно с издержками. Имеются две фирмы. Каждая производит $q_i$ товара. Функции выигрыша задается формулой
	\[
		u_i(q_1,\, q_2) = q_iP(q_1+q_2) - C_i(q_i)
	\]
	где
	\[
		P(Q) =
		\begin{cases}
			a - Q, & \text{если } Q \leq a \\
			0,	& \text{если } Q > a
		\end{cases}
	\] --- функция цены,
			$$
			C_i(q_i) = cq_i \text{ --- издержки производства}
			$$
	Параметры $a,\, c > 0$ одинаковы для обеих фирм.
	
	Будем считать, что фирмы стремятся максимизировать свою функцию выигрыша.
	
	В каком случае первая фирма будет вступать на рынок (то есть, $q_1 > 0$)?
	
	\begin{enumerate}[label=$\circ$]
		\item $q_2 > c$
		\item $q_2 < c$
		\item $q_2 < a + c$
		\item $q_2 > a + c$
		\item[$\circledcirc$] $q_2 < a - c$
		\item $q_2 > a - c$
	\end{enumerate}

	\solution
	Выигрыш первой фирмы равен $q_1(a-c-q_1-q_2)$, если $q_1+q_2 \leq a$, и $-cq_1$ иначе. Во втором случае брать $q_1 > 0$ невыгодно (можно взять $q_1=0$ и получить больше). При заданных $q_2,\, a,\, c$ максимум функции $-q_1^2 + q_1(a-c-q_2)$ положителен при $a-c-q_2 > 0$, и достигается в точке $\cfrac{a-c-q_2}{2} < a-q_2$. Поэтому первая фирма будет вступать на рынок при $q_2 < a-c$.
	
		\task
	Рассмотрим модель Курно с издержками. Имеются две фирмы. Каждая производит $q_i$ товара. Функции выигрыша задается формулой
	\[
	u_i(q_1,\, q_2) = q_iP(q_1+q_2) - C_i(q_i)
	\]
	где
	\[
	P(Q) =
	\begin{cases}
	a - Q, & \text{если } Q \leq a \\
	0,	& \text{если } Q > a
	\end{cases}
	\] --- функция цены,
	$$
	C_i(q_i) = cq_i \text{ --- издержки производства}
	$$
	Параметры $a,\, c > 0$ одинаковы для обеих фирм.
	
	Рассмотрим равновесие по Нэшу для данной игры.
	
	Пусть $a = 10,\, c = 1$. Какое значение принимает $q_2$?
	
	\textbf{Ответ}: 3
	
	\solution
	Из предыдущей задачи следует, что профиль стратегий $\left(q_1,\, q_2\right)$ будет равновесием по Нэшу, если
	\begin{align*}
	q_1 =
		\begin{cases}
		\cfrac{9 - q_2}{2}, & \text{если } q_2 \leq 9, \\
		0,	& \text{если } q2 \geq 2.
		\end{cases}
	, &&
	q_2 =
		\begin{cases}
		\cfrac{9 - q_1}{2}, & \text{если } q_2 \leq 9, \\
		0,	& \text{если } q_2 \geq 2.
	\end{cases}
	\end{align*}
	Можно нарисовать графики данных функций или решить систему уравнений и получить, что
	\[
		q_1 = q_2 = 3
	\]
	
	\task
	Рассмотрим модель Бертрана. Каждая фирма выбирает цену $p_i$. Издержки производства задаются функцией 
	$C_i (q_i) = \begin{cases}
	f + c q_i, & \text{если } q_i > 0, \\
	0, & \text{если } q_i = 0
	\end{cases}$, где $f$ -- фиксированные издержки, $c q_i$ -- издержки от производства $q_i$ единиц товара.
	
	Функция спроса $D(p) = \begin{cases}
	a - p, & \text{если } p \leq a \\
	0, & \text{если } p > a
	\end{cases}$. Параметры $a,\, f,\, c > 0$ фиксированы для всех фирм, $a > c$.
	
	Будем также считать, что если оба игрока назначили одну и ту же цену, то первый игрок производит весь товар.
	
	Функция выигрыша первого игрока $u_1$ при $p_1 \leq a$ задается следующими условиями:
	
	\begin{enumerate}[label=$\circ$]
		\item $u_1 (p_1,\, p_2) = \begin{cases}
		(p_1 - c)(a - p_1), & \text{если } p_1 \leq p_2 \\
		0, & \text{если } p_1 > p_2
		\end{cases}$
		\item[$\circledcirc$] $u_1 (p_1,\, p_2) = \begin{cases}
		(p_1 - c)(a - p_1) - f, & \text{если } p_1 \leq p_2 \\
		0, & \text{если } p_1 > p_2
		\end{cases}$
		\item $u_1 (p_1,\, p_2) = \begin{cases}
		(p_1 - c)(a - p_1), & \text{если } p_1 \leq p_2 \\
		-f, & \text{если } p_1 > p_2
		\end{cases}$
		\item $u_1 (p_1,\, p_2) = \begin{cases}
		(p_1 - c)(a - p_1) - f, & \text{если } p_1 \leq p_2 \\
		-f, & \text{если } p_1 > p_2
		\end{cases}$
	\end{enumerate}

	\solution
	Первый игрок получает выигрыш $p_1-c$ с каждого проданного товара. Количество проданных товаров равно $a-p_1$, если $p_1 \leq p_2$, и $0$, если $p_1 > p_2$. При этом у нас есть дополнительные издержки $f$ в случае производства товара (то есть в случае, когда $p_1 \leq p_2$).
	
	\task
	Рассмотрим модель Бертрана. Каждая фирма выбирает цену $p_i$. Издержки производства задаются функцией 
	$C_i (q_i) = \begin{cases}
	f + c q_i, & \text{если } q_i > 0, \\
	0, & \text{если } q_i = 0
	\end{cases}$, где $f$ -- фиксированные издержки, $c q_i$ -- издержки от производства $q_i$ единиц товара.
	
	Функция спроса $D(p) = \begin{cases}
	a - p, & \text{если } p \leq a \\
	0, & \text{если } p > a
	\end{cases}$. Параметры $a,\, f,\, c > 0$ фиксированы для всех фирм, $a > c$.
	
	Будем также считать, что если оба игрока назначили одну и ту же цену, то первый игрок производит весь товар.
	
	Пусть $a = 10,\, c = 1,\, f = 20$. Введите цену первого игрока в равновесии по Нэшу.
	
	\textbf{Ответ}: 5
	
	\solution
	Проведем рассуждения, аналогичные рассуждениям на лекции.
	
	Выигрыш любого из игроков в равновесии неотрицателен (иначе можно изменить свою стратегию и получить 0). Пусть в равновесии $p_1 \leq p_2$. Если второму игроку невыгодно заменять свою стратегию на $p_1$ или меньше, выполнено неравенство $(x-c)(a-x)-f \leq 0$ при $x < p_1$. Вкупе с предыдущим мы получаем, что $(p_1-c)(a-p_1)-f = 0$. С другой стороны, если первому игроку невыгодно менять свою стратегию, то $(x-c)(a-x)-f \leq 0$ при $x < p_2$.
	
	При данных значениях $(x-1)(10-x)-20 = 0$ в точках $x=5,\, 6$, и неравенство $(x-c)(a-x)-f \leq 0$ выполнено, только если $p_1=p_2=5$. При этом в данной точке оба игрока получают $0$, и не смогут изменить свою стратегию, чтобы получить больше.
	
	\task
	Рассмотрим следующую пространственную модель выборов. Имеется два кандидата. Мнения жителей распределены на отрезке $[0,1]$ так, что существует единственная точка $A$ (медианный избиратель), слева и справа от которой находится половина от всех жителей.
	
	Кандидаты выбирают точки на отрезке $[0,1]$, после чего каждый житель выбирает ближайшего к нему кандидата; если расстояние одинаково, то голоса делятся поровну между кандидатами. Если кандидаты выбрали одну и ту же точку, голоса также делятся поровну между ними.
	
	Кандидаты стремятся набрать больше половины голосов. Можно считать, что при выигрыше они получают $2$, при равенстве голосов $1$, при проигрыше $0$.
	
	Что можно сказать про равновесия в этой задаче?
	
	\begin{enumerate}[label=$\square$]
		\item Равновесий в этой задаче нет
		\item[$\blacksquare$] Равновесие единственно
		\item $(0.5, 0.5)$ является равновесием
		\item[$\blacksquare$] $(A,\, A)$ является равновесием
	\end{enumerate}
	
	\solution
	Докажем, что единственным равновесием будет $(A,A)$. Действительно, пусть $(x_1,\, x_2)$ --- равновесие. Какой-то из игроков не получает больше половины голосов. Тогда он может получить хотя бы половину голосов, назвав то же число, что и второй игрок. Отсюда $x_1=x_2$. Если $x_1 \neq A$, то $x_1$ может назвать $A$ и получить больше половины голосов, следовательно, $x_1=x_2=A$.
	
	\task 
	Рассмотрим следующую пространственную модель выборов. Имеется два кандидата. Мнения жителей распределены на отрезке $[0,1]$ так, что существует единственная точка $A$ (медианный избиратель), слева и справа от которой находится половина от всех жителей.
	
	Кандидаты выбирают точки на отрезке $[0,1]$, после чего каждый житель выбирает ближайшего к нему кандидата; если расстояние одинаково, то голоса делятся поровну между кандидатами. Если кандидаты выбрали одну и ту же точку, голоса также делятся поровну между ними.
	
	Изменим функцию полезности. Будем считать, что у каждого игрока есть его собственное мнение по данному вопросу: $B$ у первого игрока и $C$ у второго игрока. Функция полезности равна выигрышу от голосования (как и в предыдущей задаче мы будем говорить, что кандидаты получают $2$, если набрали больше половины голосов, $1$, если набрали ровно половину, $0$, если меньше половины голосов) минус издержки, равные расстоянию от названного числа и позиции игрока в случае его победы. То есть если второй игрок назвал $1/2$ и победил, то выигрыш равен $2-\abs{1/2-C}$.
	
	Посмотрим, как изменились равновесия в такой игре
	
	Пусть $A = 0.6,\, B = 0.3,\, C = 0.8$. Какую точку выберет первый игрок в равновесии? (Если равновесий несколько, введите любое равновесное значение. Если равновесий нет, введите $-1$)
	
	\textbf{Ответ}:	0.6
	
	\solution
	\label{week7-models:train:solution-elections}
	Найдем равновесия в этой игре. Пусть $(x_1,x_2)$ --- равновесие. Каждый игрок стремится победить и выбрать точку как можно ближе к его точке зрения. Пусть в равновесии побеждает какой-то из игроков, например, первый. Тогда ему выгоднее назвать число, которое ближе к $B$, чем его позиция. Значит, $x_1=B=0.3$. Но тогда второй назовет $0.6$, выиграет, и получит выигрыш больше (чем $0$). Аналогично с другим игроком.
	
	Если оба игрока набрали половину голосов, то каждый из них будет стремиться изменить свою стратегию и выиграть (тогда они получат больше $1$). Поэтому они будут называть число поближе к $0.6$, чем их позиция. Следовательно, $x_1=x_2=0.6$. Заметим, что при данном выборе стратегий никому из игроков не будет выгодно отклоняться. Поэтому данное состояние является единственным равновесием.
	
	\task Пусть $A = 0.6,\, B = 0.3,\, C = 0.8$. Какую точку выберет второй игрок в равновесии?
	
	(Если равновесий несколько, введите любое равновесное значение. Если равновесий нет, введите $-1$)
	
	\textbf{Ответ}: 0.6
	
	\solution См. \hyperref[week7-models:train:solution-elections]{выше}.
	
	\task
	Имеются четыре юноши (Андрей, Борис, Виктор, Георгий) и четыре девушки (Анна, Белла, Валентина, Галина). Предпочтения заданы следующим образом:
	
	\begin{tabular}{lllll}
	\hline
	Андрей: &Белла &$>$ Галина &$>$ Анна &$>$ Валентина \\
	
	Борис: &Валентина &$>$ Анна &$>$ Галина &$>$ Белла \\
	
	Виктор: &Белла &$>$ Валентина &$>$ Анна &$>$ Галина \\
	
	Георгий: &Галина &$>$ Анна &$>$ Валентина &$>$ Белла \\
	
	Анна: &Борис &$>$Андрей &$>$ Георгий &$>$ Виктор \\
	
	Белла: &Георгий &$>$ Виктор &$>$ Андрей &$>$ Борис \\
	
	Валентина:&Андрей&$>$ Георгий &$>$ Виктор &$>$ Борис \\
	
	Галина: &Борис &$>$ Андрей &$>$ Георгий &$>$ Виктор \\
	\hline
	\end{tabular}

	Какие из данных сочетаний будут устойчивыми к разрыву?
	
	\begin{enumerate}[label=$\square$]
		\item (Андрей, Анна), (Борис, Белла), (Виктор, Валентина), (Георгий, Галина)
		\item[$\blacksquare$] (Андрей, Галина), (Борис, Анна), (Виктор, Белла), (Георгий, Валентина)
		\item (Андрей, Валентина), (Борис, Галина), (Виктор, Анна), (Георгий, Белла)
		\item (Андрей, Белла), (Борис, Валентина), (Виктор, Галина), (Георгий,Анна)
	\end{enumerate}

	\solution
	Явная угроза устойчиовости - это пара (юноша, девушка), которые хорошо друг друга ценят. К примеру, Анна считает Бориса лучше всех, а Борис считает Анну второй по предочтению. Следовательно, если Борис не с Анной и не с Валентиной, то обязательно образуется пара (Анна, Борис).
	
	В случае, когда Борис с Валентиной (4-й вариант), для Валентины это худший выбор. Попробуем найти ей пару получше. Отсюда находим пару (Валентина, Виктор), для обоих игроков этот выбор лучше, чем предлагаемый.
	
	Осталось рассмотреть сочетание (Андрей, Галина), (Борис, Анна), (Виктор, Белла), (Георгий, Анна). Для Виктора это лучший выбор. Андрей поменяется только если Белла согласится, но Белла считает Виктора лучше, чем Андрей. Борис поменяется только если Валентина согласится, но она считает его хуже всех. Наконец, Георгий может предложить разорвать брак Галине, но она считает Андрей лучше. Поэтому данное сочетание будет устойчивым.
	
	\task
		Имеются четыре юноши (Андрей, Борис, Виктор, Георгий) и четыре девушки (Анна, Белла, Валентина, Галина). Предпочтения заданы следующим образом:
	
	\begin{tabular}{lllll}
		\hline
		Андрей: &Белла &$>$ Галина &$>$ Анна &$>$ Валентина \\
		
		Борис: &Валентина &$>$ Анна &$>$ Галина &$>$ Белла \\
		
		Виктор: &Белла &$>$ Валентина &$>$ Анна &$>$ Галина \\
		
		Георгий: &Галина &$>$ Анна &$>$ Валентина &$>$ Белла \\
		
		Анна: &Борис &$>$Андрей &$>$ Георгий &$>$ Виктор \\
		
		Белла: &Георгий &$>$ Виктор &$>$ Андрей &$>$ Борис \\
		
		Валентина:&Андрей&$>$ Георгий &$>$ Виктор &$>$ Борис \\
		
		Галина: &Борис &$>$ Андрей &$>$ Георгий &$>$ Виктор \\
		\hline
	\end{tabular}

	Воспользуемся алгоритмом Гейла-Шепли. Сначала посмотрим, что получится, если юноши делают предпочтение девушкам, а потом наоборот и сравним результаты.
	
	Применим алгортим ГШ, если юноши делают предложения, а девушки отказывают. Найдите паросочетание и выпишите заглавные буквы имён девушек в алфавитном порядке юношей (то есть ВГАБ означает, что Андрею досталась Валентина, Борису - Галина, Виктору - Анна, Георгию - Белла) без пробелов и кавычек.
	
	\textbf{Ответ}: ГВБА
	
	\solution См. \hyperref[week7-models:train:marriage]{ниже}.
	
	\task
	Применим алгортим ГШ, если девушки делают предложения, а юноши отказывают. Найдите паросочетание и выпишите заглавные буквы имён девушек в алфавитном порядке юношей (то есть ВГАБ означает, что Андрею досталась Валентина, Борису - Галина, Виктору - Анна, Георгию - Белла) без пробелов и кавычек.
	
	\textbf{Ответ}: ГАБВ
	
	\solution См. \hyperref[week7-models:train:marriage]{ниже}.
	
	\task
	Сравним полученные результаты. В каком случае юношам лучше: когда они предлагают, или когда они отказывают?
	\begin{enumerate}[label=$\circ$]
		\item[$\circledcirc$] Лучше, когда юноши предлагают
		\item Лучше, когда юноши отказывают
		\item Нельзя сказать: кому-то лучше, кому-то хуже
	\end{enumerate}

	\solution
	\label{week7-models:train:marriage}
	Будем записывать алгоритм в виде этапов, на каждом этапе запишем кто кому сделал предложение.
	
	Юноши предлагают:
	
	1-й этап: Белле: Андрей,Виктор, Валентине: Борис; Галине: Георгий
	
	Белла отказывает Андрею
	
	2-й этап Белле: Виктор, Валентине: Борис; Галине: Андрей,Георгий
	
	Галина отказывает Георгию
	
	3-й этап Анне: Георгий, Белле: Виктор, Валентине: Борис; Галине: Андрей
	
	Паросочетание найдено. \textit{Ответ}: ГВБА
	
	Девушки предлагают:
	
	1-й этап: Андрею: Валентина Борису: Анна, Галина; Георгию: Белла
	
	Борис отказывает Галине, она делает предложение Андрею
	
	2-й этап: Андрею: Валентина, Галина Борису: Анна; Георгию: Белла
	
	Андрей отказывает Валентине, она делает предложение Георгию
	
	3-й этап: Андрею: Галина, Борису: Анна; Георгию: Белла, Валентина
	
	Георгий отказывает Белле, она делает предложение Виктору
	
	3-й этап: Андрею: Галина, Борису: Анна; Виктору: Белла, Георгию: Валентина
	
	Паросочетание найдено. \textit{Ответ}: ГАБВ
	
	Сравнив полученные результаты, получаем, что Борису в первом случае досталась Валентина, что лучше Анны, а Георгию Анна, которая лучше Валентины. Поэтому юношам выгоднее делать предложения, чем отказывать. Несложно показать, что и в общем случае те, кто делают предложение, получают лучший для себя исход.
	\section{Неделя 6}
	
	\subsection{Тренировочные задачи}
	
	\subsubsection{Задача 1}
	Имеется 5 продавцов и 3 покупателя. У каждого продавца есть по 1 товару. Все товары одинаковы. Каждый из покупателей хочет приобрести 1 единицу товара. Если $m$ продавцов и $n$ покупателей встречаются, то их полезность равна количеству проданных единиц товара, то есть $\min (m, n)$.
	
	Обозначим продавцов через $A,\,B,\,C,\,D,\,E$, а покупателей через $1,\,2,\,3$. Формализуем данную игру как коалиционную.
	
	Какие из приведенных ниже выигрышей коалиций заданы правильно?
	
	\subsubsection*{Решение}
	Необходимо лишь проверить, что выигрыш в точности равен $v(m, n) = \min(m, n)$. Этому условию удовлетворяет два ответа:
	\begin{enumerate}[label=-]
		\item $v(A, B, 1, 2) = 2$
		\item $v(A, B, 1, 2, 3) = 2$
	\end{enumerate}
		
	\subsubsection{Задача 2}
	Имеется 5 продавцов и 3 покупателя. У каждого продавца есть по 1 товару. Все товары одинаковы. Каждый из покупателей хочет приобрести 1 единицу товара. Если $m$ продавцов и $n$ покупателей встречаются, то их полезность равна количеству проданных единиц товара, то есть $\min(m,n)$.
	
	Обозначим продавцов через $A,\,B,\,C,\,D,\,E$, а покупателей через $1,\,2,\,3$. Формализуем данную игру как коалиционную.
	
	Рассмотрим ядро этой игры. Какие утверждения про ядро выполнены?
	
	\begin{enumerate}[label=$\circ$]
		\item Ядро пусто.
		\item Ядро состоит из единственного дележа: продавцы получают по 1/5, покупатели - по 2/3.
		\item[$\circledcirc$] Ядро состоит из единственного дележа: продавцы получают по 0, покупатели - по 1.
		\item Ядро состоит из всех платежей, в которых $x_1+x_2+x_3 = 3$ (здесь $x_i$ - платёж покупателя $i$).
	\end{enumerate}

	\subsubsection*{Решение}
Выпишем условие на ядро. Имеется равенство $x_{A}+x_{B}+x_{C}+x_{D}+x_{E}+x_{1}+x_2+x_3=3$ и набор неравенств для всех коалиций. Заметим, что $x_A+x_B+x_C +x_1+x_2+x_3 \geq 3$, откуда $x_D+x_E=0$, а значит, $x_D=x_E=0$. Аналогично, $x_A=x_B=x_C=0$. Из неравенства $x_A+x_i\geq 1$ для любого $i$ получаем, что $x_1, x_2, x_3 \geq 1$, а так как их сумма равна 3, то $x_1=x_2=x_3=1$.

Легко проверить что данный делёж удовлетворяет всем неравенствам.
	
	\subsubsection{Задача 3}
	Поменяем количество продавцов и покупателей в предыдущей задаче. Несложно понять, что такое же ядро будет и в игре $M$ продавцов и $N$ покупателей при $M \neq N$. Поймём что будет, если их количества совпадают. Рассмотрим следующую игру.
	
	Имеется 4 продавца и 4 покупателя. У каждого продавца есть по 1 товару. Все товары одинаковы. Каждый из покупателей хочет приобрести 1 единицу товара. Если mm продавцов и nn покупателей встречаются, то их полезность равна количеству проданных единиц товара, то есть $\min(m,n)$.
	
	Обозначим продавцов через $A,B,C,D$, а покупателей через $1,2,3,4$. Формализуем данную игру как коалиционную.
	
	Рассмотрим ядро этой игры. Какие дележи входят в ядро?
	
	\begin{enumerate}[label=$\circledcirc$]
		\item Всем продавцам - 0, всем покупателям - по 1.
		\item Всем продавцам - 1, всем покупателям - по 0.
		\item Всем игрокам по 1/2.
		\item Всем продавцам по 2/3, всем покупателям - по 1/3.
	\end{enumerate}

	\subsubsection*{Решение}
	Выпишем условие на ядро. Имеется равенство $x_{A}+x_{B}+x_{C}+x_{D}+x_{1}+x_2+x_3+x_4=4$ и набор неравенств для всех коалиций. Заметим, что $x_{A}+x_1\geq 1$, $x_B+x_2 \geq 1$, $x_C+x_3\geq 1$, $x_D+x_4\geq 1$. Так как сумма переменных в правой части неравенств равна 4, мы получаем, что данные неравенства являются равенствами. Отсюда несложно получить, что $x_A=x_B=x_C=x_D=\alpha$, $x_1=x_2=x_3=x_4=1-\alpha$. При этом данные дележи удовлетворяют всем неравенствам.
	
	Следовательно, все дележи из условия подходят.

	\subsubsection{Задача 4}
	\label{sec4}
	Трое жителей: Аркадий, Борис и Василий, живут в соседних домиках. У каждого из них есть предпочтение в каком из домиков жить (эти предпочтения заданы в виде полезности). Несколько жителей могут договориться поменяться домиками для максимизации суммарной полезности. Предположим, что полезности заданы в виде следующей таблицы (по строчкам жители, а по столбцам домики):
	
	\begin{table}[h]
		\label{prob4:table1}
		\centering
		\begin{tabular}{|c|c|c|c|}
			\hline & А & Б & В \\ 	
			\hline Аркадий & 2 & 4 & 7 \\ 
			\hline Борис & 3 & 2 & 1 \\ 
			\hline Василий & 5 & 4 & 4 \\ 
			\hline 
		\end{tabular} 
	\end{table}
	
	Формализуем данную игру как коалиционную (занумеруем игроков в алфавитном порядке). Рассмотрим ядро данной игры.
	
	Отметьте верные неравенства, накладываемые на дележ $(x_1,x_2,x_3)$ из ядра игры (которые соответствуют условиям, накладываемым на коалиции).
	
	\begin{enumerate}[label=$\circ$]
		\item $x_1 \geq 7$
		\item[$\circledcirc$] $x_1 + x_2 \geq 7$
		\item[$\circledcirc$] $x_2 + x_3 \geq 6$
		\item $x_1 + x_3 \geq 11$
	\end{enumerate}

	\subsubsection*{Решение}
	Аркадий сам по себе не может поменять свой домик. Поэтому выигрыш коалиции $A$ равен стоимости домика для Аркадия, то есть $2$. Поэтому соответствующее неравенство должно выглядеть так: $x_1 \geq 2$.
	
	Рассмотрим Аркадия и Бориса. Если они не поменяются домиками, их суммарная полезность будет равна $2+2=4$, а если поменяются, то $4+3=7$. Отсюда выигрыш из коалиции составляет 7, и поэтому $x_1+x_2 \geq 7$.
	
	Рассмотрим Бориса и Василия. Если они не поменяются домиками, их суммарная полезность будет равна $2+4=6$, а если поменяются, то $1+4=5$. Отсюда выигрыш из коалиции составляет 6, и поэтому $x_2+x_3 \geq 6$.
	
	Рассмотрим Аркадия и Василия. Если они не поменяются домиками, их суммарная полезность будет равна $2+4=6$, а если поменяются, то $7+5=12$. Отсюда выигрыш из коалиции составляет 12, и поэтому $x_1+x_3 \geq 12$.

	\subsubsection{Задача 5}
	См. условие задачи \hyperref[sec4]{\textbf{4}}.
	
	Пусть $x_1,x_2,x_3$ -- дележ из ядра, $x_1 = 7.5$. Введите любое $x_2$ (платеж Бориса), которое подходит под данное условие. (Если при данном $x_1$ не существует дележа, лежащего в ядре, впишите $-1$).
	
	\textbf{Ответ:} 2 (см. \hyperref[sec6:sol]{\textit{ниже}})
	
	\subsubsection{Задача 6}
	\label{sec6}
	Пусть $x_1,~x_2,~x_3$ -- дележ из ядра, $x_1 = 7.5$. Введите любое $x_3$ (платеж Василия), которое подходит под данное условие. (Если при данном $x_1$ не существует дележа, лежащего в ядре, впишите $-1$).
	
	\subsubsection*{Решение}
	\label{sec6:sol}
	Заметим, что $x_1+x_2+x_3=1$, $x_1+x_3 \geq 12$, $x_2 \geq 2$. Отсюда следует, что $x_2=2$ (если ядро не пусто). Найдём условия на $x_1,~x_3$. Получаем систему неравенств:
	\[
	\left\{
		\begin{aligned}
			& x_1 \geq 2, \\
			& x_3 \geq 4, \\
			& x_1 + x_3 = 12, \\
			& x_1 + x_2 \geq 7, \\
			& x_2 + x_3 \geq 6,
		\end{aligned}
	\right.
	\Rightarrow
	\left\{
		\begin{aligned}
			& x_1 \geq 5, \\
			& x_3 \geq 4, \\
			& x_1 + x_3 = 12
		\end{aligned}
	\right.
	\Rightarrow
	\left\{
		\begin{aligned}
			& x_1 = 5 + t, \\
			& x_3 = 7 - t, \\
			& t \in [0,\, 3]
		\end{aligned}
	\right.
	\]

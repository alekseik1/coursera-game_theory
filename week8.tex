\section{Финал}

\subsection{Экзамен}

\task
В следующих вопросах требуется отметить все верные утверждения. Внимание, правильных ответов может быть больше одного! Ответ "да" подразумевает, что данное утверждение верно для всех игр из описываемого класса. Ответ "нет" подразумевает, что найдётся игра, в которой данное утверждение выполнено не будет.

Дана игра двух лиц в нормальной форме, у каждого игрока есть по две стратегии. Тогда

\begin{enumerate}[label=$\square$]
	%\item[$\blacksquare$] Существует равновесие по Нэшу в чистых стратегиях.
	%\item[$\blacksquare$] Если в результате вычеркивания строго доминируемых стратегий остаётся ровно один исход, то он является равновесием по Нэшу.
	%\item[$\blacksquare$] 	Если в результате вычеркивания строго доминируемых стратегий остаётся ровно один исход, то никакие другие исходы не являются равновесием по Нэшу.
	%\item[$\blacksquare$] Если в результате вычеркивания слабо доминируемых стратегий остаётся ровно один исход, то он является равновесием по Нэшу.
	%\item Если в результате вычеркивания слабо доминируемых стратегий остаётся ровно один исход, то никакие другие исходы не являются равновесием по Нэшу.
	% todo: проверить ниже
	%\item[$\blacksquare$] Существует равновесие по Нэшу в чистых стратегиях.
	%\item[$\blacksquare$] Если в результате вычеркивания строго доминируемых стратегий остаётся ровно один исход, то он является равновесием по Нэшу.
	%\item Если в результате вычеркивания строго доминируемых стратегий остаётся ровно один исход, то никакие другие исходы не являются равновесием по Нэшу.
	%\item[$\blacksquare$] Если в результате вычеркивания слабо доминируемых стратегий остаётся ровно один исход, то он является равновесием по Нэшу.
	%\item Если в результате вычеркивания слабо доминируемых стратегий остаётся ровно один исход, то никакие другие исходы не являются равновесием по Нэшу.
	
	% todo: проверить ниже
	%\item[$\blacksquare$] Существует равновесие по Нэшу в чистых стратегиях.
	%\item[$\blacksquare$] Если в результате вычеркивания строго доминируемых стратегий остаётся ровно один исход, то он является равновесием по Нэшу.
	%\item Если в результате вычеркивания строго доминируемых стратегий остаётся ровно один исход, то никакие другие исходы не являются равновесием по Нэшу.
	%\item Если в результате вычеркивания слабо доминируемых стратегий остаётся ровно один исход, то он является равновесием по Нэшу.
	%\item Если в результате вычеркивания слабо доминируемых стратегий остаётся ровно один исход, то никакие другие исходы не являются равновесием по Нэшу.
	
	% todo: проверить ниже
	%\item[$\blacksquare$] Существует равновесие по Нэшу в чистых стратегиях.
	%\item[$\blacksquare$] Если в результате вычеркивания строго доминируемых стратегий остаётся ровно один исход, то он является равновесием по Нэшу.
	%\item Если в результате вычеркивания строго доминируемых стратегий остаётся ровно один исход, то никакие другие исходы не являются равновесием по Нэшу.
	%\item Если в результате вычеркивания слабо доминируемых стратегий остаётся ровно один исход, то он является равновесием по Нэшу.
	%\item Если в результате вычеркивания слабо доминируемых стратегий остаётся ровно один исход, то никакие другие исходы не являются равновесием по Нэшу.
	
	% todo: проверить ниже
	%\item[$\blacksquare$] Существует равновесие по Нэшу в чистых стратегиях.
	%\item[$\blacksquare$] Если в результате вычеркивания строго доминируемых стратегий остаётся ровно один исход, то он является равновесием по Нэшу.
	%\item[$\blacksquare$] Если в результате вычеркивания строго доминируемых стратегий остаётся ровно один исход, то никакие другие исходы не являются равновесием по Нэшу.
	%\item Если в результате вычеркивания слабо доминируемых стратегий остаётся ровно один исход, то он является равновесием по Нэшу.
	%\item Если в результате вычеркивания слабо доминируемых стратегий остаётся ровно один исход, то никакие другие исходы не являются равновесием по Нэшу.
	
	%\item[$\blacksquare$] Существует равновесие по Нэшу в чистых стратегиях.
	%\item Если в результате вычеркивания строго доминируемых стратегий остаётся ровно один исход, то он является равновесием по Нэшу.
	%\item[$\blacksquare$] Если в результате вычеркивания строго доминируемых стратегий остаётся ровно один исход, то никакие другие исходы не являются равновесием по Нэшу.
	%\item[$\blacksquare$] Если в результате вычеркивания слабо доминируемых стратегий остаётся ровно один исход, то он является равновесием по Нэшу.
	%\item Если в результате вычеркивания слабо доминируемых стратегий остаётся ровно один исход, то никакие другие исходы не являются равновесием по Нэшу.
	
	% 5 ВАРИАНТ
	
	%\item[$\blacksquare$] Существует равновесие по Нэшу в чистых стратегиях.
	%\item Если в результате вычеркивания строго доминируемых стратегий остаётся ровно один исход, то он является равновесием по Нэшу.
	%\item Если в результате вычеркивания строго доминируемых стратегий остаётся ровно один исход, то никакие другие исходы не являются равновесием по Нэшу.
	%\item Если в результате вычеркивания слабо доминируемых стратегий остаётся ровно один исход, то он является равновесием по Нэшу.
	%\item[$\blacksquare$] Если в результате вычеркивания слабо доминируемых стратегий остаётся ровно один исход, то никакие другие исходы не являются равновесием по Нэшу.
	
	%\item Существует равновесие по Нэшу в чистых стратегиях.
	%\item[$\blacksquare$] Если в результате вычеркивания строго доминируемых стратегий остаётся ровно один исход, то он является равновесием по Нэшу.
	%\item Если в результате вычеркивания строго доминируемых стратегий остаётся ровно один исход, то никакие другие исходы не являются равновесием по Нэшу.
	%\item Если в результате вычеркивания слабо доминируемых стратегий остаётся ровно один исход, то он является равновесием по Нэшу.
	%\item[$\blacksquare$] Если в результате вычеркивания слабо доминируемых стратегий остаётся ровно один исход, то никакие другие исходы не являются равновесием по Нэшу.
	
	%\item Существует равновесие по Нэшу в чистых стратегиях.
	%\item Если в результате вычеркивания строго доминируемых стратегий остаётся ровно один исход, то он является равновесием по Нэшу.
	%\item[$\blacksquare$] Если в результате вычеркивания строго доминируемых стратегий остаётся ровно один исход, то никакие другие исходы не являются равновесием по Нэшу.
	%\item Если в результате вычеркивания слабо доминируемых стратегий остаётся ровно один исход, то он является равновесием по Нэшу.
	%\item[$\blacksquare$] Если в результате вычеркивания слабо доминируемых стратегий остаётся ровно один исход, то никакие другие исходы не являются равновесием по Нэшу.
	
	% todo: проверить ниже
	%\item Существует равновесие по Нэшу в чистых стратегиях.
	%\item Если в результате вычеркивания строго доминируемых стратегий остаётся ровно один исход, то он является равновесием по Нэшу.
	%\item Если в результате вычеркивания строго доминируемых стратегий остаётся ровно один исход, то никакие другие исходы не являются равновесием по Нэшу.
	%\item[$\blacksquare$] Если в результате вычеркивания слабо доминируемых стратегий остаётся ровно один исход, то он является равновесием по Нэшу.
	%\item[$\blacksquare$] Если в результате вычеркивания слабо доминируемых стратегий остаётся ровно один исход, то никакие другие исходы не являются равновесием по Нэшу.
	
	% todo: проверить ниже
	%\item[$\blacksquare$] Существует равновесие по Нэшу в чистых стратегиях.
	%\item[$\blacksquare$] Если в результате вычеркивания строго доминируемых стратегий остаётся ровно один исход, то он является равновесием по Нэшу.
	%\item Если в результате вычеркивания строго доминируемых стратегий остаётся ровно один исход, то никакие другие исходы не являются равновесием по Нэшу.
	%\item Если в результате вычеркивания слабо доминируемых стратегий остаётся ровно один исход, то он является равновесием по Нэшу.
	%\item[$\blacksquare$] Если в результате вычеркивания слабо доминируемых стратегий остаётся ровно один исход, то никакие другие исходы не являются равновесием по Нэшу.
	
	%\item[$\blacksquare$] Существует равновесие по Нэшу в чистых стратегиях.
	%\item Если в результате вычеркивания строго доминируемых стратегий остаётся ровно один исход, то он является равновесием по Нэшу.
	%\item[$\blacksquare$] Если в результате вычеркивания строго доминируемых стратегий остаётся ровно один исход, то никакие другие исходы не являются равновесием по Нэшу.
	%\item Если в результате вычеркивания слабо доминируемых стратегий остаётся ровно один исход, то он является равновесием по Нэшу.
	%\item[$\blacksquare$] Если в результате вычеркивания слабо доминируемых стратегий остаётся ровно один исход, то никакие другие исходы не являются равновесием по Нэшу.
	
	% todo: проверить ниже
	%\item[$\blacksquare$] Существует равновесие по Нэшу в чистых стратегиях.
	%\item[$\blacksquare$] Если в результате вычеркивания строго доминируемых стратегий остаётся ровно один исход, то он является равновесием по Нэшу.
	%\item Если в результате вычеркивания строго доминируемых стратегий остаётся ровно один исход, то никакие другие исходы не являются равновесием по Нэшу.
	%\item[$\blacksquare$] Если в результате вычеркивания слабо доминируемых стратегий остаётся ровно один исход, то он является равновесием по Нэшу.
	%\item[$\blacksquare$] Если в результате вычеркивания слабо доминируемых стратегий остаётся ровно один исход, то никакие другие исходы не являются равновесием по Нэшу.
	
	
	
	%%%%%%%%%%%
	
	\item Существует равновесие по Нэшу в чистых стратегиях.
	\item Если в результате вычеркивания строго доминируемых стратегий остаётся ровно один исход, то он является равновесием по Нэшу.
	\item Если в результате вычеркивания строго доминируемых стратегий остаётся ровно один исход, то никакие другие исходы не являются равновесием по Нэшу.
	\item Если в результате вычеркивания слабо доминируемых стратегий остаётся ровно один исход, то он является равновесием по Нэшу.
	\item[$\blacksquare$] Если в результате вычеркивания слабо доминируемых стратегий остаётся ровно один исход, то никакие другие исходы не являются равновесием по Нэшу.
	
\end{enumerate}

\task
Дана игра двух лиц в нормальной форме, у каждого игрока есть по две стратегии. Тогда
\begin{enumerate}[label=$\square$]	% тут все норм
	\item[$\blacksquare$] Существует равновесие по Нэшу в смешанных стратегиях.
	\item Если в этой игре есть единственное равновесие по Нэшу в чистых стратегиях, то и в смешанных стратегиях равновесие будет только одно.
	\item[$\blacksquare$] Существует игра, в которой имеется бесконечное количество смешанных равновесий по Нэшу.
	\item[$\blacksquare$] Если в игре есть сильно доминируемая стратегия, то она входит в смешанное равновесие с нулевой вероятностью.
	\item Если в игре есть слабо доминируемая стратегия, то она входит в смешанное равновесие с нулевой вероятностью.
\end{enumerate}

\task
Дана симметричная игра двух лиц в нормальной форме, у каждого игрока по две стратегии. Тогда

\begin{enumerate}[label=$\square$]
	%\item Существует симметричное смешанное равновесие по Нэшу.
	%\item Не существует несимметричного смешанного равновесия по Нэшу.	
	%\item[$\blacksquare$] Если в игре есть сильно доминируемая стратегия, то существует единственное равновесие по Нэшу в чистых стратегиях.
	%\item Если в игре есть сильно доминируемая стратегия, то существует единственное равновесие по Нэшу в смешанных стратегиях.
	
	% todo: проверить ниже
	%\item Существует симметричное смешанное равновесие по Нэшу.
	%\item[$\blacksquare$] Не существует несимметричного смешанного равновесия по Нэшу.	
	%\item Если в игре есть сильно доминируемая стратегия, то существует единственное равновесие по Нэшу в чистых стратегиях.
	%\item Если в игре есть сильно доминируемая стратегия, то существует единственное равновесие по Нэшу в смешанных стратегиях.
	
	% todo: проверить ниже
	%\item Существует симметричное смешанное равновесие по Нэшу.
	%\item[$\blacksquare$] Не существует несимметричного смешанного равновесия по Нэшу.	
	%\item[$\blacksquare$] Если в игре есть сильно доминируемая стратегия, то существует единственное равновесие по Нэшу в чистых стратегиях.
	%\item Если в игре есть сильно доминируемая стратегия, то существует единственное равновесие по Нэшу в смешанных стратегиях.
	
	% todo: проверить ниже
	%\item Существует симметричное смешанное равновесие по Нэшу.
	%\item Не существует несимметричного смешанного равновесия по Нэшу.	
	%\item[$\blacksquare$] Если в игре есть сильно доминируемая стратегия, то существует единственное равновесие по Нэшу в чистых стратегиях.
	%\item Если в игре есть сильно доминируемая стратегия, то существует единственное равновесие по Нэшу в смешанных стратегиях.
	
	% todo: проверить ниже
	%\item[$\blacksquare$] Существует симметричное смешанное равновесие по Нэшу.
	%\item Не существует несимметричного смешанного равновесия по Нэшу.	
	%\item Если в игре есть сильно доминируемая стратегия, то существует единственное равновесие по Нэшу в чистых стратегиях.
	%\item Если в игре есть сильно доминируемая стратегия, то существует единственное равновесие по Нэшу в смешанных стратегиях.
	
	%\item[$\blacksquare$] Существует симметричное смешанное равновесие по Нэшу.
	%\item Не существует несимметричного смешанного равновесия по Нэшу.	
	%\item[$\blacksquare$] Если в игре есть сильно доминируемая стратегия, то существует единственное равновесие по Нэшу в чистых стратегиях.
	%\item Если в игре есть сильно доминируемая стратегия, то существует единственное равновесие по Нэшу в смешанных стратегиях.
	
	%\item[$\blacksquare$] Существует симметричное смешанное равновесие по Нэшу.
	%\item[$\blacksquare$] Не существует несимметричного смешанного равновесия по Нэшу.	
	%\item Если в игре есть сильно доминируемая стратегия, то существует единственное равновесие по Нэшу в чистых стратегиях.
	%\item Если в игре есть сильно доминируемая стратегия, то существует единственное равновесие по Нэшу в смешанных стратегиях.
	
	%\item[$\blacksquare$] Существует симметричное смешанное равновесие по Нэшу.
	%\item[$\blacksquare$] Не существует несимметричного смешанного равновесия по Нэшу.	
	%\item[$\blacksquare$] Если в игре есть сильно доминируемая стратегия, то существует единственное равновесие по Нэшу в чистых стратегиях.
	%\item Если в игре есть сильно доминируемая стратегия, то существует единственное равновесие по Нэшу в смешанных стратегиях.
	
	% 4 ВАРИАНТ
	
	% todo: проверить ниже
	%\item Существует симметричное смешанное равновесие по Нэшу.
	%\item Не существует несимметричного смешанного равновесия по Нэшу.	
	%\item Если в игре есть сильно доминируемая стратегия, то существует единственное равновесие по Нэшу в чистых стратегиях.
	%\item[$\blacksquare$] Если в игре есть сильно доминируемая стратегия, то существует единственное равновесие по Нэшу в смешанных стратегиях.
	
	% todo: проверить ниже
	%\item[$\blacksquare$] Существует симметричное смешанное равновесие по Нэшу.
	%\item Не существует несимметричного смешанного равновесия по Нэшу.	
	%\item Если в игре есть сильно доминируемая стратегия, то существует единственное равновесие по Нэшу в чистых стратегиях.
	%\item[$\blacksquare$] Если в игре есть сильно доминируемая стратегия, то существует единственное равновесие по Нэшу в смешанных стратегиях.
	
	% todo: проверить ниже
	%\item Существует симметричное смешанное равновесие по Нэшу.
	%\item[$\blacksquare$] Не существует несимметричного смешанного равновесия по Нэшу.	
	%\item Если в игре есть сильно доминируемая стратегия, то существует единственное равновесие по Нэшу в чистых стратегиях.
	%\item[$\blacksquare$] Если в игре есть сильно доминируемая стратегия, то существует единственное равновесие по Нэшу в смешанных стратегиях.
	
	%\item Существует симметричное смешанное равновесие по Нэшу.
	%\item Не существует несимметричного смешанного равновесия по Нэшу.	
	%\item[$\blacksquare$] Если в игре есть сильно доминируемая стратегия, то существует единственное равновесие по Нэшу в чистых стратегиях.
	%\item[$\blacksquare$] Если в игре есть сильно доминируемая стратегия, то существует единственное равновесие по Нэшу в смешанных стратегиях.
	
	%\item[$\blacksquare$] Существует симметричное смешанное равновесие по Нэшу.
	%\item[$\blacksquare$] Не существует несимметричного смешанного равновесия по Нэшу.	
	%\item Если в игре есть сильно доминируемая стратегия, то существует единственное равновесие по Нэшу в чистых стратегиях.
	%\item[$\blacksquare$] Если в игре есть сильно доминируемая стратегия, то существует единственное равновесие по Нэшу в смешанных стратегиях.
	
	%%%%%%%%%%%%%%%%%%%
	
	\item[$\blacksquare$] Существует симметричное смешанное равновесие по Нэшу.
	\item Не существует несимметричного смешанного равновесия по Нэшу.	
	\item[$\blacksquare$] Если в игре есть сильно доминируемая стратегия, то существует единственное равновесие по Нэшу в чистых стратегиях.
	\item[$\blacksquare$] Если в игре есть сильно доминируемая стратегия, то существует единственное равновесие по Нэшу в смешанных стратегиях.
\end{enumerate}

\task
Рассмотрим коалиционную игру с побочными платежами. Тогда

\begin{enumerate}[label=$\square$]	% тут все норм
	\item[$\blacksquare$] Если игра супермодулярна, то вектор Шепли лежит в ядре.
	\item Если вектор Шепли лежит в ядре, то игра супермодулярна.
	\item Если ядро непусто, то вектор Шепли лежит в ядре.
	\item Если ядро непусто, то игра супермодулярна.
	\item[$\blacksquare$] Если игра супермодулярна, то ядро непусто.
\end{enumerate}

\task
\underline{АУКЦИОН} \\

Рассмотрим следующий вариант аукциона. Двое борются за приз стоимостью $10$. Игра ведётся следующим образом: вначале у первого игрока уже задана ставка $1$. Он может либо отказаться от участия: тогда никто ничего не платит и приз достаётся второму игроку. Если он принимает участие, то ход переходит ко второму игроку. Далее на каждом ходе игрок может либо повысить свою ставку на 1, либо отказаться от участия. Игра заканчивается, когда один из игроков отказывается от участия. В этом случае оба игрока платят свою последнюю названную ставку, а приз достаётся тому, кто не отказался. Если оба игрока всё время повышают ставки, выплаты обоим равны $(-\infty,-\infty)$.

Пример проведения такого аукциона: первый игрок остаётся в игре (ставки $1$ --- у первого игрока, $0$ --- у второго), второй повышает $[1,1]$, первый повышает $[2,1]$, второй повышает $[2,2]$, первый отказывается. Итог: приз достаётся второму, оба платят ставки, то есть выигрыши равны $(-2,8)$.

Рассмотрим игру как игру в развернутой форме.

Какие из перечисленных профилей являются равновесием в данной игре?

\begin{enumerate}[label=$\square$]
	%\item[$\blacksquare$] Первый всегда отказывается от игры, второй всё время повышает.
	%\item[$\blacksquare$] Первый всё время повышает, второй всё время отказывается от игры.
	%\item Оба игрока всё время повышают до бесконечности.
	%\item[$\blacksquare$] Оба игрока повышают ставки до 10, а дальше всё время отказываются.
	
	% todo: проверить ниже
	%\item Первый всегда отказывается от игры, второй всё время повышает.
	%\item[$\blacksquare$] Первый всё время повышает, второй всё время отказывается от игры.
	%\item Оба игрока всё время повышают до бесконечности.
	%\item[$\blacksquare$] Оба игрока повышают ставки до 10, а дальше всё время отказываются.
	
	% todo: проверить ниже
	%\item Первый всегда отказывается от игры, второй всё время повышает.
	%\item Первый всё время повышает, второй всё время отказывается от игры.
	%\item Оба игрока всё время повышают до бесконечности.
	%\item[$\blacksquare$] Оба игрока повышают ставки до 10, а дальше всё время отказываются.
	
	% todo: проверить ниже
	%\item[$\blacksquare$] Первый всегда отказывается от игры, второй всё время повышает.
	%\item[$\blacksquare$] Первый всё время повышает, второй всё время отказывается от игры.
	%\item[$\blacksquare$] Оба игрока всё время повышают до бесконечности.
	%\item[$\blacksquare$] Оба игрока повышают ставки до 10, а дальше всё время отказываются.
	
	% todo: проверить ниже
	%\item[$\blacksquare$] Первый всегда отказывается от игры, второй всё время повышает.
	%\item Первый всё время повышает, второй всё время отказывается от игры.
	%\item Оба игрока всё время повышают до бесконечности.
	%\item[$\blacksquare$] Оба игрока повышают ставки до 10, а дальше всё время отказываются.
	
	% todo: проверить ниже
	%\item Первый всегда отказывается от игры, второй всё время повышает.
	%\item Первый всё время повышает, второй всё время отказывается от игры.
	%\item[$\blacksquare$] Оба игрока всё время повышают до бесконечности.
	%\item[$\blacksquare$] Оба игрока повышают ставки до 10, а дальше всё время отказываются.
	
	% todo: проверить ниже
	%\item[$\blacksquare$] Первый всегда отказывается от игры, второй всё время повышает.
	%\item Первый всё время повышает, второй всё время отказывается от игры.
	%\item[$\blacksquare$] Оба игрока всё время повышают до бесконечности.
	%\item[$\blacksquare$] Оба игрока повышают ставки до 10, а дальше всё время отказываются.
	
	%\item Первый всегда отказывается от игры, второй всё время повышает.
	%\item[$\blacksquare$] Первый всё время повышает, второй всё время отказывается от игры.
	%\item[$\blacksquare$] Оба игрока всё время повышают до бесконечности.
	%\item[$\blacksquare$] Оба игрока повышают ставки до 10, а дальше всё время отказываются.
	
	% 4 ТОЧНО НЕВЕРЕН
	
	% 3 ВАРИАНТ
	
	%\item Первый всегда отказывается от игры, второй всё время повышает.
	%\item Первый всё время повышает, второй всё время отказывается от игры.
	%\item[$\blacksquare$] Оба игрока всё время повышают до бесконечности.
	%\item Оба игрока повышают ставки до 10, а дальше всё время отказываются.
	
	%\item[$\blacksquare$] Первый всегда отказывается от игры, второй всё время повышает.
	%\item Первый всё время повышает, второй всё время отказывается от игры.
	%\item[$\blacksquare$] Оба игрока всё время повышают до бесконечности.
	%\item Оба игрока повышают ставки до 10, а дальше всё время отказываются.
	
	% todo: проверить ниже
	%\item Первый всегда отказывается от игры, второй всё время повышает.
	%\item[$\blacksquare$] Первый всё время повышает, второй всё время отказывается от игры.
	%\item[$\blacksquare$] Оба игрока всё время повышают до бесконечности.
	%\item Оба игрока повышают ставки до 10, а дальше всё время отказываются.
	
	%\item[$\blacksquare$] Первый всегда отказывается от игры, второй всё время повышает.
	%\item[$\blacksquare$] Первый всё время повышает, второй всё время отказывается от игры.
	%\item[$\blacksquare$] Оба игрока всё время повышают до бесконечности.
	%\item Оба игрока повышают ставки до 10, а дальше всё время отказываются.
	
	%%%%%%%%%%%%
	
	% 3 ТОЧНО НЕВЕРЕН
	
	% 2 ВАРИАНТ
	
	%\item Первый всегда отказывается от игры, второй всё время повышает.
	%\item[$\blacksquare$] Первый всё время повышает, второй всё время отказывается от игры.
	%\item Оба игрока всё время повышают до бесконечности.
	%\item Оба игрока повышают ставки до 10, а дальше всё время отказываются.
	
	%%%%%%%%%%%%
	
	% 1 ВАРИАНТ
	
	%\item[$\blacksquare$] Первый всегда отказывается от игры, второй всё время повышает.
	%\item Первый всё время повышает, второй всё время отказывается от игры.
	%\item Оба игрока всё время повышают до бесконечности.
	%\item Оба игрока повышают ставки до 10, а дальше всё время отказываются.
	
	\item[$\blacksquare$] Первый всегда отказывается от игры, второй всё время повышает.
	\item[$\blacksquare$] Первый всё время повышает, второй всё время отказывается от игры.
	\item Оба игрока всё время повышают до бесконечности.
	\item Оба игрока повышают ставки до 10, а дальше всё время отказываются.
\end{enumerate}

\task
Предположим, что оба игрока используют смешанную стратегию следующего вида: на каждом шаге первый отказывается от дальнейшей игры с вероятностью $p$, второй отказывается от дальнейшей игры с вероятностью $q$. При этом $p,q \neq 0,1$. Введите любое значение $p$, при котором такая стратегия будет являться равновесной. Если такого $p$ не существует, введите $-1$.

\textbf{Ответ}: % ВЕРНО
1/10

\task
\underline{ПАРЛАМЕНТ} \\

Конгресс и Сенат состоят из трех членов каждый. Закон принимается только если в обеих палатах набрано большинство.

Формализуем данную игру как кооперативную (закон принят - 1, не принят - 0).

Отметьте верные утверждения:

\begin{enumerate}[label=$\square$]
	% ВЕРНО
	\item[$\blacksquare$] Ядро данной игры пусто.
	\item Игра супермодулярна.
	\item Вектор Шепли лежит в ядре.
	\item[$\blacksquare$] В векторе Шепли все координаты равны между собой.
\end{enumerate}

\task
Предположим теперь, что есть ещё и президент, одобрение которого обязательно. Какой вес он будет иметь в векторе Шепли?

\textbf{Ответ}: 
%$1/7$ % todo: проверить
%2/7
%3/7
%4/7
%5/7 % todo: проверить
%6/7
%5/21
%2/6
%3/6
%4/6
%5/6
%5/9
%7/9
%1/9
2/9

\task
А потом две палаты объединили. Теперь нужно просто 4 голоса из 6 плюс президентское одобрение. Вроде бы, парламент стал сильнее, так как больше выигрывающих коалиций, чем в предыдущей задаче. Посмотрим, как изменилась зарплата президента. Какой вес он будет иметь в векторе Шепли?

\textbf{Ответ}: 
3/7  % ВЕРНО

\task
\underline{РАВНОВЕСИЕ ВАЛЬРАСА}

У Аркадия 5 апельсинов и 3 банана, у Бориса - 6 апельсинов и 4 банана. Функция полезности Аркадия - произведение количества имеющихся у него фруктов, фукция полезности Борис - минимум фруктов (то есть при 4 апельсинах и 3 бананах полезность Аркадия - 12, Бориса - 3)

Рассмотрим распределения товаров, удовлетворяющих свойствам T1-T3' (такое перераспределение имеющихся продуктов, которое удовлетворяет свойствам коалиционной устойчивости, Парето-эффективности, неухудшаемости и индивидуалной рациональности). Напишите максимальное количество апельсинов, которое может достаться Борису в таких распределениях с точностью до 0.01. Если таких распределений нет, введите -1.

\textbf{Ответ}: 
%4 % todo: проверить
%5
%6
%7
%8
%3
%9
%10
%11
%12
%13
%14
%15
%13.32
%3.14

\task
Рассмотрим равновесный вектор цен $(1,p_B)$, то есть вектор цен, удовлеторяющий свойствам I-IV (выдаём каждому деньги вместо продуктов, дальше каждый покупает такое количество продуктов, чтобы максимизировать свою полезность и получается, что продуктов хватило ровно на всех). Какое максимальное значение принимает $p_B$? Ответ введите с точностью до 0.01. Если такого вектора нет, введите $-1$.

\textbf{Ответ}: 
%0.56 % todo: проверить
%0.61
%0.25
%0.77
%0.73
%1.23
%-1
%1.52
%2.21
%2.22
%2.23
%3.14
%2.12
%2.15

\task
\underline{БОНУСНЫЙ ВОПРОС}

Как зовут лектора?

\begin{enumerate}[label=$\circ$]
	\item Дарт Вейдер
	\item[$\circledcirc$] Алексей Владимирович
	\item Санта Клаус
	\item Владимир Иванович
\end{enumerate}